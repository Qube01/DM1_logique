\documentclass{article}

% Language setting
% Replace `english' with e.g. `spanish' to change the document language
\usepackage[T1]{fontenc}
\usepackage[french]{babel}

% Set page size and margins
% Replace `letterpaper' with`a4paper' for UK/EU standard size
\usepackage[letterpaper,top=2cm,bottom=2cm,left=3cm,right=3cm,marginparwidth=1.75cm]{geometry}

% Useful packages
\usepackage{float}
\usepackage{amsfonts}
\usepackage{amsmath}
\usepackage{graphicx}
\usepackage{listings} % import code
\usepackage[colorlinks=true, allcolors=blue]{hyperref}


\title{Logique et preuves - DM1}
\author{LOLUM Théophane, \\PERDRIZET Corentin, \\PERIN Côme}

\begin{document}
\maketitle

\section{}

\text{La formalisation des dépendances et conflits en logique propositionnelle donne :}

\begin{align*}
    (a\implies(b\land (c\lor d)\land(d \lor e)\land(d \lor f))) \\
    \land(b\implies g)\land(c\implies(g\lor h\lor i))\land(c\implies \neg e) \\
    \land(d\implies h\lor i)\land(e \implies\neg c)\land(e\implies\neg i) \\
    \land(e\implies j)\land(f\implies j)\land(g\implies\neg h)
\end{align*}

\lstset{language=Verilog}

Le code du fichier \texttt{fig4.smt} :
    \begin{lstlisting}
(declare-const a Bool)
(declare-const b Bool)
(declare-const c Bool)
(declare-const d Bool)
(declare-const e Bool)
(declare-const f Bool)
(declare-const g Bool)
(declare-const h Bool)
(declare-const i Bool)
(declare-const j Bool)

(assert (implies a (and b (or c d) (or d e) (or d f))))
(assert (implies b g))
(assert (implies c (or g h i)))
(assert (implies c (not e)))
(assert (implies d (or h i)))
(assert (implies e (not c)))
(assert (implies e (not i)))
(assert (implies e j))
(assert (implies f j))
(assert (implies g (not h)))
    \end{lstlisting}
    
pour spécifier à Z3 quels sont les packages actuellement installés il faut ajouter les \texttt{assert} correspondant parmi les lignes précédentes.
    
\textit{exemple : }\\
Si le package p est installé, on ajoute : \texttt{(assert p)}


\section{}
Le code du fichier \texttt{ex2.smt} : 
\begin{lstlisting}
(assert h)
(assert b)

(check-sat)
\end{lstlisting}

On possède désormais une formule propositionnelle logique satisfiable si et seulement si le package b peut être installé lorsque h est installé.

\lstset{language=bash}
\begin{lstlisting}
    $ cat fig4.smt ex2.smt | z3 -smt2 -in
    unsat
\end{lstlisting}

La formule est bien insatisfiable.


\section{}
\section{}
\section{}
\section{}
\section{}


\end{document}
